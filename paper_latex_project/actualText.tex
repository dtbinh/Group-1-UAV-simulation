\section{Introduction}
Unmanned Aerial Vehicles (UAV) are gaining popularity in private as well as commercial sector. One often-mentioned use-case is the parcel delivery. In the near future, UAV might be used to deliver parcels to your doorstep. The logistics industry hopes for positive impacts on their costs and competitiveness. The deployment of a delivery concept is quite complex. To allow for early evaluation and reduce the number of required flight-testing, a simulation engine is required. This is why SIMULATIONNAME has been developed. It supports multiple depots and UAV as well as  simple change of path-planning algorithm. UAVs move in a 3D environment  The engine does not simulate the behaviour of the UAV in the aerial space itself but allows for evaluation of spatial distribution of depots or the suitability of a path-planning algorithm. It was developed considering the use of swarm algorithms in the delivery use-case and allow for inter-drone communication and autonomous organisation avoiding a central ground control.

\section{Related Work}
Literature on simulation engines for UAV delivery is still relatively scarce. Simulation engines that were developed concentrate rather on specific components or flight behaviour.
\cite{johnson.2001} developed a simulation engine for the behaviour of specific hardware components of a UAV. \cite{lu.2011} simulated the flight dynamic behaviour of a UAV using Matlab/Simulink. MultiUAV \cite{rasmussen.2003}, a very general simulation engine has been developed to simulate cooperative algorithms for finding targets. In 2005,  an agent-based simulation engine implementing a surveillance scenario has been developed \cite{jang.2005}. UAV try to find targets without prior knowledge of the targets’ locations. They implement UAVs as agents and also use the notion of obstacles and bases. The sensor concept is similar to the one used in this paper.

\section{Scenario}
Before embarking the software model, the delivery scenario shall be described.  Every UAV has a depot, in the engine called basestation, from which it receives items to be delivered. An item's destination is within the range of a Basestation which means that every Basestation has a limited range in the area. UAVs are assigned to one specific Basestation and only deliver items of it. After receiving an item, the UAV will begin flying to the specific destination. It scans the area and stores information on all fields it has seen en-route. If a UAV meets another one, they will exchange information on the already-explored parts of the area.\\
If a field on the route contains an obstacle, it will deviate from its route and try to find one around the obstacle. Obstacles can have different heights, thus UAV als\\
After delivering the item, it will fly back to the basestation to receive a new item. If the battery reaches a certain threshold, the UAV will fly to the nearest basestation (not neccessarily its home base) and recharge its battery. 
 UAVs are moving in a 3D environment which means that obstacles can have different heights.

\section{Description of the model}
\subsection{MESA}
MESA \cite{masad.2015} is an agent-based modeling framework (ABM) developed specifically for the Python programming language. 
\begin{enumerate}
	\item Agent-Based, shortly describe what this means
	\item MESA concepts in general \cite{masad.2015}
	\begin{enumerate}
		\item Agent-Based Modeling framework for Python in particular as it got a more important language over time
		\item Agent schedule, 
		\item visualization in a browser-based interface to avoid additional dependencies to GUI frameworks
		\item allows for data analysis using Python's built-in analysis tools
		\item Activation: Calling step method of agent, leads to an action. In every step, all agents are activated. We are using random activation: Random order of agent activations. Why did we take this? Find a stupid reason
		\item Space is implemented using Grid. A grid has rectangular fields, used because this is closest to reality, every grid can contain multiple agents (different heights). Grid stores the agents and has methods to add, move or remove them
		\item DataCollector component is used to collect data and easily export 
		
	\end{enumerate}
	\item Which components did we remove or modify and why?
	\item Reprogramming of rendering engine. How and why?
	
	\item Why did we decide for specific components or not?
\end{enumerate}
\subsection{Simulation Model}
\begin{enumerate}
	
	\item General Structure 
	
\end{enumerate}
\subsection{UAV}
\begin{enumerate}
	\item Perceived World
	\item Components/Sensors (make a drawing)
	\item States
\end{enumerate}


\cite{jang.2005} can be helpful to structure this part!


\begin{enumerate}
	\item multiple depots
	\item Describe the modular and extensible character of the architecture
	\item Describe how the model accomodates for the integration of self-organizing algorithms for UAVs
	\item Describing changes and extensions in the mesa framework
	\item drones have to recharge and thus deviate from their actual route
	\item ...
	
\end{enumerate}


\subsection{Algorithm}
To provide an example simulation, the A* algorithm has been used 
SOURCES!

\section{Evaluation}
\begin{enumerate}
	\item Use KPI's that we defined
	\item Compare with existing
	%\item Use graphics with cool Excel tables to present data or we make tables here
\end{enumerate}

\section{Conclusion}
\begin{enumerate}
	\item Overall summary
	\item limitations of this approach
	\item future development?
\end{enumerate}

